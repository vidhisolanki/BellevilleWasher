\documentclass[a4paper]{article}

\usepackage[margin=1in]{geometry} % full-width

% AMS Packages
\usepackage{amsmath}
\usepackage{amsthm}
\usepackage{amssymb}

% Unicode
\usepackage[utf8]{inputenc}
\usepackage[hyphens]{url}
\usepackage{hyperref}
\hypersetup{breaklinks=true}
\urlstyle{same}


% Natbib
\bibliographystyle{ieeetr}

\usepackage{graphicx, color}
\graphicspath{{Figures/}}

% Author info
\title{Effectiveness of Belleville Washers for Seismic Retrofit of Substation Equipment}
\author{Author One$^1$\thanks{Author One was partially supported by Grant XXX} \and Author Two$^2$ \and Author Three$^1$}

%\date{
%	$^1$Organization 1 \\ \texttt{\{auth1, auth3\}@org1.edu}\\%
%	$^2$Organization 2 \\ \texttt{auth3@inst2.edu}\\[2ex]%
%	\today
%}

\begin{document}
	\maketitle
	
\begin{abstract}
Seattle City Light (SCL), a power utility in the Pacific Northwest where there is potential for large earthquakes, has retrofitted several Capacitive Voltage Transformers (CVT) in their substations with Belleville washer arrangements at the bases to improve seismic performance. The strategy is simple and relatively low-cost to implement, and preliminary testing at SCL has demonstrated improved seismic performance. The washer arrangements essentially function as seismic isolators, reducing the frequency and increasing damping. To expand the implementation of this strategy more broadly for other equipment and at other utilities’ substations, this project systematically characterizes the stiffness and damping mechanisms of Belleville washer arrangements and tests a retrofitted 230 kV CVT on an earthquake simulator. Special instrumentation is developed to monitor the mechanics of the Belleville washer arrangements including force, deformation and equipment base moment measurements. A detailed modeling and analysis procedure is developed, allowing for either a nonlinear dynamic approach or an equivalent linear dynamic approach, to predict the response of equipment such as CVTs outfitted with Belleville washer stacks for seismic protection. The analysis procedure has been validated with experimental measurements. Essentially, knowing the force-deformation response of a washer configuration, obtained for example by cyclic loading in a material testing machine, together with the inertia characteristics of the equipment, the seismic response of the equipment can be predicted. This will enable washer configurations to be designed and deployed readily for seismic protection of various equipment.
\vspace{\baselineskip}	

\noindent\textbf{Keywords:} Substation equipment, seismic performance improvement, Belleville washers/disc springs, nonlinear and linear dynamic analysis, damping, experimental validation
\end{abstract}

	
\section{Introduction}	\label{sec:intro}

Base isolation and supplemental damping systems are now widely used as seismic protective 
systems in structures. They represent a design philosophy that seeks to reduce seismic demand 
on functional structural components, and to minimize if not eliminate damage and permanent 
deformation even under large earthquakes. This is in contrast to the conventional seismic design 
philosophy, wherein structural components are designed to be ductile, and undergo permanent 
deformation without failure. For critical structures and equipment, such as in electrical 
substations, the approach of reducing seismic demand and permanent deformations is more 
attractive to ensure continued functioning of such structures and equipment post-earthquake. The 
concept of frequency modification for substation equipment was recommended as early as 1973 
in a report prepared for Bonneville Power Administration (BPA) \cite{couch1973procedures}. Base isolation and 
supplemental damping systems are being increasingly deployed in substations for seismic 
protection \cite{cochran2015seismic, Kempner2015isolation, oikonomou2016seismic, saadeghvaziri2009seismic}.\\

One approach to modifying frequency and adding damping is using stacks of Belleville washers. 
Belleville washers, named after Julian Belleville, who patented them in 1867, are circular 
washers given a taper to form a conical shape (Figure --). They are also known as a disc 
springs or conical spring washers, although the term “disc springs” may be more appropriate \cite{NBlunt}
for the type of dynamic application considered in this report. When a compressive load is applied 
along the axis, the washer flattens by way of elastic deformation, resulting in spring action \cite{shigley2004standard}. 
When washers are nested such that the bottom surface of one is placed on the top surface of the 
next, frictional rubbing between these surfaces results in hysteretic behavior (see Chapter 3), and
hence energy dissipation. The washers can also be stacked in different ways as illustrated in 
Figure 1 and Table 1-2. Thus the stiffness and dissipation of a stack of Belleville washers can 
be tuned. Testing discussed further in Chapter 3 has shown that the force-displacement hysteresis 
of washer stacks is stable and repeatable. These stacks therefore lend themselves to modeling, 
and use as engineered frequency modification and energy dissipation devices.
For early analyses of the mechanical behavior of Belleville washers, see \cite{ashworth1946disk, almen1936uniform}, considering both 
single washer and nesting. For a more recent study on the analysis of washer mechanical 
behavior, see \cite{zhou2023modeling} and the references therein. The force-displacement behavior of a single washer 
is given by \cite{shigley2004standard, almen1936uniform}
%\bibliographystyle{IEEEtran}
\bibliography{References}
	
\end{document} 