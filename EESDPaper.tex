\documentclass[a4paper]{article}

\usepackage[margin=1in]{geometry} % full-width

% AMS Packages
\usepackage{amsmath}
\usepackage{amsthm}
\usepackage{amssymb}

% Unicode
\usepackage[utf8]{inputenc}
\usepackage{hyperref}

% Natbib
\bibliographystyle{ieeestr}

\usepackage{graphicx, color}
\graphicspath{{Figures/}}

% Author info
\title{Effectiveness of Belleville Washers for Seismic Retrofit of Substation Equipment}
\author{Author One$^1$\thanks{Author One was partially supported by Grant XXX} \and Author Two$^2$ \and Author Three$^1$}

%\date{
%	$^1$Organization 1 \\ \texttt{\{auth1, auth3\}@org1.edu}\\%
%	$^2$Organization 2 \\ \texttt{auth3@inst2.edu}\\[2ex]%
%	\today
%}

\begin{document}
	\maketitle
	
\begin{abstract}
Seattle City Light (SCL), a power utility in the Pacific Northwest where there is potential for large earthquakes, has retrofitted several Capacitive Voltage Transformers (CVT) in their substations with Belleville washer arrangements at the bases to improve seismic performance. The strategy is simple and relatively low-cost to implement, and preliminary testing at SCL has demonstrated improved seismic performance. The washer arrangements essentially function as seismic isolators, reducing the frequency and increasing damping. To expand the implementation of this strategy more broadly for other equipment and at other utilities’ substations, this project systematically characterizes the stiffness and damping mechanisms of Belleville washer arrangements and tests a retrofitted 230 kV CVT on an earthquake simulator. Special instrumentation is developed to monitor the mechanics of the Belleville washer arrangements including force, deformation and equipment base moment measurements. A detailed modeling and analysis procedure is developed, allowing for either a nonlinear dynamic approach or an equivalent linear dynamic approach, to predict the response of equipment such as CVTs outfitted with Belleville washer stacks for seismic protection. The analysis procedure has been validated with experimental measurements. Essentially, knowing the force-deformation response of a washer configuration, obtained for example by cyclic loading in a material testing machine, together with the inertia characteristics of the equipment, the seismic response of the equipment can be predicted. This will enable washer configurations to be designed and deployed readily for seismic protection of various equipment.
\vspace{\baselineskip}	

\noindent\textbf{Keywords:} Substation equipment, seismic performance improvement, Belleville washers/disc springs, nonlinear and linear dynamic analysis, damping, experimental validation
\end{abstract}

	
\section{Introduction}	\label{sec:intro}

\bibliographystyle{IEEEtran}
\bibliography{../References}
	
\end{document}